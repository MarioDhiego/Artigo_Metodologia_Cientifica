
\documentclass[12pt]{article}
\usepackage{sbc-template}
\usepackage{graphicx,url}
\usepackage{xcolor}
\usepackage[brazil]{babel}   
\usepackage[utf8]{inputenc}
\usepackage[portuguese]{babel}
\graphicspath{ {./} }
\usepackage{quoting}
\usepackage{hyphenat}
\usepackage{hyperref}
\usepackage{multicol}

\sloppy
\title{Mensuração do Desempenho (\textit{proficiência}) dos Estudantes do Curso de Bacharelado em Sistemas de Informação, na Prova do ENADE/2017 via Teoria de Resposta ao Item}

\author{Mário D. R. Valente\inst{1}, Dionne C. Monteiro\inst{2}, Heliton R. Tavares \inst{3}}

\address{Graduando em Sistemas de Informação - Universidade Federal do Pará(UFPA) \\
  Rua. Augusto Corrêa 01 -- Guamá -- Belém -- PA -- Brasil
\nextinstitute
  Professor Dr., Faculdade de Computação - Universidade Federal do Pará(UFPA)\\
  Av. Augusto Corrêa 01 -- Guamá -- Belém - -- PA -- Brasil 
  \nextinstitute
Professor Dr., Faculdade de Estatística - Universidade Federal do Pará(UFPA)\\
  Av. Augusto Corrêa 01 -- Guamá -- Belém - -- PA -- Brasil 
    \email{diego.vatente@gmail.com, dionne@ufpa.br, Heliton@ufpa.br}}


\begin{document} 

\maketitle

\begin{abstract}
 This study aimed to measure the performance (proficiency) of Computing students enrolled at UFPA, who took the Enade through Item Response Theory (IRT). The measurement of student performance was carried out using the tree-parameter logistic model (ML3P), with data extracted from the website of the National Institute of Educational Studies and Research Anísio Teixeira (Inep) related to the Enade test in 2017, covering 235 students . The elaborated scale allowed to discriminate the students in three levels of performance. The results also showed that the items contained in the ENADE test represented a high degree of difficulty for the group that took the test, in which UFPA students have low performance in the subjects of Artificial Intelligence, Computing Theory, and Database. This result suggests that actions by HEI and public policies capable of contributing to improving the performance presented by students are necessary.
  
\noindent
  \textbf{Keywords}: Performance, Proficiency, ENADE, Item Response Theory.
\end{abstract}

\begin{resumo} 
Este estudo teve como objetivo mensurar o desempenho (proficiência) dos estudantes de Sistemas de Informação matriculados na UFPA, que realizaram o Enade via Teoria de Resposta ao Item (TRI). A mensuração do desempenho dos estudantes foi realizada, utilizando-se o modelo logístico de três parâmetros (ML3P), com dados extraídos do site do (Inep) relativos à prova Enade em 2017, contemplando 11.503 estudantes. A escala elaborada permitiu discriminar os estudantes em três níveis de desempenho. Os resultados apontaram também que os itens contidos na prova ENADE representaram um alto grau de dificuldade para o grupo que realizou a prova, no qual os alunos da UFPA possuem baixo desempenho nos temas de Probabilidade, Programação, Inteligência Artificial, Teoria da Computação, e Banco de Dados. Este resultado sugere que são necessárias ações da IES e políticas públicas capazes de contribuir para melhoria do desempenho apresentado pelos estudantes.
  
\noindent
  \textbf{Palavras-chave}: Desempenho, Proficiência, ENADE, Teoria de Resposta ao Item.
\end{resumo}

\newpage
\section{Introdução}

O Exame Nacional de Desempenho dos Estudantes (ENADE) é um dos pilares da avaliação dos Sistema Nacional de Avaliação da Educação Superior (SINAES), criado pela Lei nº 10.861, de 14 de abril de 2004. 

O ENADE é uma prova aplicada anualmente aos alunos (ingressantes e concluintes) de cursos de graduação, em áreas definidas pelo Ministério da Educação e é realizado pelo INEP. O Enade é atualmente, o único instrumento oficial que avalia os resultados do processo de ensino aprendizagem e tem um peso significativo no desempenho das Instituições de Nível Superior no Brasil(BRASIL, 2017).

A Prova do ENADE (2017), com duração total de 4 (quatro) horas, apresentou um componente de avaliação na Formação Geral (FG), comum aos cursos de todas as áreas e um Componente Específico (CE) de Cada Área. 

No componente de avaliação da Formação Geral, foram consideradas as seguintes características integrantes do perfil profissional, como sua formação ética, humanista e voltadas ao exercício da cidadania. A parte referente ao Componente Específico contribui com 75\% da nota final, enquanto a parcela, referente a Formação Geral, contribui com 25\%, em consonância com o número de questões da prova, 30 e 10, respectivamente. Os conceitos utilizados no ENADE variam de 1 a 5, e, à medida que esse valor aumenta, melhor terá sido o desempenho no exame.

A nota final do estudante no Enade é obtida pela média ponderada na qual a parte de Formação Geral responde por 25\%, e a parte de conhecimento específico, por 75\%. A nota de Conhecimento Específico é a média ponderada das duas notas, Objetiva e Discursiva, com pesos iguais a, respectivamente, 15\% e 85\% e arredondada a uma casa decimal.

Desta forma o presente trabalho tem o objetivo de Mensurar o Desempenho (proficiência) dos Estudantes nos Cursos de Sistemas de Informação da UFPA na Prova do ENADE em 2017, por meio da Teoria de Resposta ao Item (TRI), indicando perspectivas de uso dos resultados para o aprimoramento do Projeto Político Pedagógico da Faculdade de Computação.


\section{Trabalho Correlatos} \label{sec:firstpage}
No Brasil, alguns trabalhos abordaram a Teoria da Resposta ao Item (TRI) para avaliação de proficiência com dados do ENADE. Oliveira (2006) utilizou a TRI para proceder a uma análise das propriedades psicométricas da prova do ENADE de 2004, aplicada aos alunos dos cursos de medicina. Nogueira (2008) aplicou a TRI para avaliar as questões da prova de formação geral do ENADE de 2004 e 2005, a fim de estimar a proficiência dos estudantes e o ajuste dos itens ao modelo de Rasch.

 Primi, Hutz e Silva (2011) realizaram análise psicométrica dos itens empregando os modelos Rasch e de créditos parciais na prova do ENADE de Psicologia de 2006.

Camargo et al. (2016) mensuraram a proficiência dos estudantes do curso de Ciências Contábeis no ENADE utilizando TRI, demonstrando que os itens da prova representaram um alto grau de dificuldade para os estudantes e, em geral, os estudantes apresentaram proficiências muito baixas.

\newpage
Figueiró e Mozzaquatro (2017) analisaram os microdados do Enade de 2014, para avaliar o desempenho dos estudantes do curso de Ciência da
Computação do Estado do Rio Grande do Sul, bem como foram feitas comparações para
agrupar as instituições públicas e privadas do mesmo estado de acordo com seus
desempenhos.


Em 2018, Lima et al., desenvolveram uma metodologia para classificação das questões do ENADE em temas. Araujo(2019), desenvolveu uma ferramenta web com dados de todos os cursos do Enade 2017, utilizando algoritimo de machine learning, para classificar as notas dos alunos por desempenho.

Charão et al. (2020) avaliaram o desempenhos dos alunos de Ciência da Computação da UFSM, na prova do Enade (2005 a 2017), explorando os resultados para subsidiar e reunir contribuições para a revisão do projeto pedagógico do curso. 

Cunha et al. (2021) desenvolveram uma ferramenta de análise de dados automática com os microdados do Enade (2005 a 201) para o curso de Ciência da Computação da UFPA, via linguagem de programação Python.

Rendeiro et al, (2022) analisou o desenpenho dos estudantes de engenharia da computação da UFPA com base nos dados do Enade (2014,2017 e 2019).




\section{Material e Métodos}
\subsection{População e Coleta de Dados}

A população do estudo é constituída pelos alunos regularmente matriculados nos cursos de graduação em Sistemas de Informação, que fizeram a prova do ENADE em 2017. 

Foi realizado o download dos microdados com extensão .txt por meio do site do INEP, e as respostas corretas receberam o valor 1(um) e as respostas erradas foram substituídas por 0(zero). Para o estudo foram consideradas, exclusivamente, as questões objetivas, ou seja 35(trinta e cinco) questões (oito de formação geral e vinte e sete de conhecimento específico).

\subsection{Recursos Computacionais}
\label{sec:metbet}

Para mensuração do desempenho (proficiência) dos estudantes no ENADE foi implementada um script no software R-Project versão 4.4 (R DEVELOPMENT CORE TEAM, 2020) e um ambiente de desenvolvimento integrado chamado Rstudio versão 1.1.47 com uso dos pacotes: mirt (CHALMERS, 2012), ltm (RIZOPOULOS, 2006), irtoys (PARTCHEV, 2010), utilizando um Modelo Logístico de 3 parâmetros (ML3P) via Teoria de Resposta.

\subsection{Classificação dos Itens por Conteúdos Curriculares}

Para realizar análises sobre um histórico de provas, é necessário adotar uma classificação comum para os conteúdos curriculares abordados nas 35 questões objetivas da prova do Enade. No quais os conteúdos são baseados nas diretrizes curriculares nacionais para os cursos de graduação, mas a cada edição estes são agrupados de forma diferente pelas comissões assessoras do INEP. Com isso, as cinco edições do exame para os cursos na area de computação, houve variação no agrupamento dos conteúdos de formação espefícica: 15 tópicos (2005 e 2008), 21 em (2011) e 17 em 2014 e 2017 (CHARÃO et al., 2020).

Para escolha de uma classificação para os itens, foi utilizado como referência o trabalho de Charão et al, (2020), tomando-se por referência as 17 categorias de conteúdos de formação específica do último exame em 2017 e 2014. Para os exames de 2005, 2008 e 2011, um especialista realizou manualmente a categorização de cada questão em até três temas, com base nas categorias do ENADE do 2017. 

\subsubsection{Análise dos Dados}

A mensuração do desempenho foi desenvolvida por meio da Teoria da Resposta ao Item (TRI), e A estimação dos parâmetros foi realizada pelo Método de Máxima Verossimilhança Marginal, com a aplicação conjunta de um processo interativo chamado de algoritmo Newton-Raphson, conforme Andrade et al. (2000). Para a comparações entre os modelos gerados utilizou-se o Teste da Razão de Verossimilhança por meio da Anova e os critérios AIC (Akaike Information Criterion) e BIC (Bayesian Information Criterion).

\newpage
\section{RESULTADOS E DISCUSSÕES}
  
Nos resultados obtidos pelos respondentes neste teste, observa-se notas variando entre 0 e 30 acertos, de um total de 35 itens. Destaca-se que ocorreu escore nulo (nenhum acertos, somente 284 alunos), como também não existiram respondentes que obtiveram escore total.  


\begin{table}[htb]
	\centering
	\caption{Medidas Estatísticas dos \textbf{Escore Bruto Geral} para os itens da Prova do Enade/2017 referentes aos alunos matriculados nos Cursos de Sistemas de Informação no Brasil (\textbf{n=11.523})}.
\begin{tabular}{l|c|c|c}
\hline\hline 
 Escores(0 a 35)  & FG      &  CE     & Escore Bruto   \\
\hline\hline
 Nº de Itens      & 8       &  27     &   35           \\
 Média            &  3.99   &  10.90  &   14.90        \\
 Mínimo           &  0      &  0      &   0            \\
 Máximo           &  8      &  23     &  30            \\
 1º Quartil(25\%) &  3      &  8      &   12           \\
 3º Quartil(75\%) &  5      &  13     &   18           \\
\hline\hline
\end{tabular}
\end{table}


\begin{table}[htb]
	\centering
	\caption{Medidas Estatísticas referentes as \textbf{Nota Bruta Geral} para os itens da Prova do Enade/2017 referentes aos alunos matriculados nos curso de Sistemas de Informação no Brasil (\textbf{n=11.523}).}
\begin{tabular}{l|c|c|c}
\hline\hline 
    Notas (0 a 100)           &    FG    &   CE  &  Nota Bruta  \\
\hline\hline
%Nº de Alunos       &  11.523 &                     \\
% \% Fez Prova      &         &                      \\
 Mínimo            &  0      &  0                   &  0           \\
 Máximmo           &  100    &  95.5                &  87.30       \\
 Média             & 49.95   &  46.71               &  44.65       \\
 1º Quartil(25\%)  &  37.50  &  36.40               &  35.50       \\
 Mediana           &  50.0   & 45.50                &  44.50       \\
 3º Quartil(75\%)  &  62.50  &  59.10               &  53.70       \\
\hline\hline
\end{tabular}
\end{table}
\vskip0.3cm





O processo de estimação dos itens foi realizado em vários passos, estratégia utilizada para manter o maior número possível de itens na prova do ENADE. A qualidade dos itens foi avaliada considerando-se, principalmente, os valores referentes às estimativas dos parâmetros de discriminação e de dificuldade e, ainda, os erros-padrão (EP) destas estimativas.

Depois do primeiro passo, ou seja, após a primeira calibração (estimação) sem a retirada de itens, os parâmetros dos itens foram sendo novamente re-estimados por meio do programa R, que estima conjuntamente os itens e coloca a habilidade numa mesma escala, fazendo com que todos os escores dos alunos sejam comparáveis, por meio do processo de equalização.

A tabela x apresenta os parâmetros de discriminação ($a_{i}$), dificuldade ($b_{i}$), e acerto ao acaso ($c_{i}$) dos itens avaliados. Tais parâmetros foram estimados no software $R_{4.4}$ com escala 0,1.


\begin{table}[]
	\centering
	\caption{Estimação dos Parâmetros dos Itens para o Modelo Logístico(ML3PL) da TRI para a Prova do Enade/2017 no Curso de Sistemas de Informação do Brasil (\textbf{n=11.523}).}
\begin{tabular}{c|l|c|c|c}
\hline\hline 
  Itens     &  Conteúdo                                            &  $(a_{i})$  & $(b_{i})$ & $(c_{i})$ \\
\hline\hline
  Q01       &  Globalização e Política Internacional               &             &           &     \\
  Q02       &  Vida Urbana/Rural/Trabalho;CTS.                     &             &           &       \\
  Q03       &  Meio Ambiente;Responsabilidade Social               &             &           &       \\
  Q04       &  TIC/Ciência/Tecnologia/Sociedade.                   &             &           &     \\
  Q05       & Inovação Tecnológica/Meio Ambiente                   &             &           &       \\
  Q06       & Processos Migratórios;Relações/Trabalho              &             &           &       \\
  Q07       & Cultura e Arte                                       &             &           &      \\
  Q08       & Responsabilidade Social/Ética e Cidadania            &             &           &       \\
  Q09       & Algoritmos/Estruturas de Dados                       &             &           &    \\
  Q10       & Engenharia de Software/Interação Homem-Computador.   &             &           &     \\
  Q11       & Paradigmas de Linguagens de Programação.             &             &           &       \\
  Q12       & Arquitetura de Computadores/Sistemas Operacionais.   &             &           &       \\
  Q13       & Sistemas Digitais.                                   &             &           &       \\
  Q14       & Lógica e Matemática Discreta.                        &             &           &       \\
  Q15       & Redes de Computadores                                &             &           &      \\
  Q16       & Ética, Computador e Sociedade.                       &             &           &      \\
  Q17       & Inteligência Artificial e Computacional.             &             &           &      \\
  Q18       & Algoritmos/Estruturas de Dados.                      &             &           &       \\
  Q19       & Banco de Dados.                                      &             &           &       \\
  Q20       & Redes de Computadores.                               &             &           &      \\
  Q21       & Lógica e Matemática Discreta.                        &             &           &       \\
  Q22       & Fundamentos e Técnicas de Programação.               &             &           &    \\
  Q23       & Teoria da Computação.                                &             &           &     \\
  Q24       & Teoria dos Grafos/Algoritmos/Estruturas de Dados.    &             &           &     \\
  Q25       & Teoria da Computação/Algoritmos/Estruturas de Dados. &             &           &     \\
  Q26       & Computação Gráfica; Processamento de Imagem.         &             &           &     \\
  Q27       & Computação Gráfica; Processamento de Imagem.         &             &           &    \\
  Q28       & Engenharia de Software/Interação Homem-Computador.   &             &           &    \\
  Q29       & Arquitetura de Computadores/Sistemas Operacionais.   &             &           &     \\
  Q30       & Compiladores.                                        &             &           &   \\
  Q31       & Sistemas Distribuídos.                               &             &           &    \\
  Q32       & Inteligência Artificial e Computacional.             &             &           &    \\
  Q33       & Teoria da Computação/Algoritmos/Estruturas de Dados. &             &           &     \\
  Q34       & Banco de Dados                                       &             &           &     \\
  Q35       & Arquitetura de Computadores/Sistemas Operacionais    &             &           &    \\
\hline\hline
\end{tabular}
\end{table}
\vskip0.3cm


\newpage
Os parâmetros $a_{i}$ indica a discriminação de cada item, quanto maior o valor deste parâmetro, maior é o seu poder de discriminação. Em outras palavras, indica o quanto cada item consegue diferenciar aqueles indivíduos que possuem o conhecimento avaliado daqueles que não o possuem. Os itens com maior poder de discriminação foram os itens: xxxxx. 

O parâmetros $b_{i}$ representa a dificuldade de cada item, quanto maior seu valor, maior a proficiência necessária dos estudantes para responder a questão. Conforme a tabela X, os itens que possuem maior grau de dificuldade para os estudantes avaliados foram: xxxxxxx. Por outro lado, os itens mais fáceis para os estudantes foram: xxxxxxx.

Depois de terminada a fase de calibração dos parâmetros dos itens, é feita a estimação das habilidades dos respondentes. De acordo com a tabela 8 verificou-se que, os itens 9, 10, 14, 25 e 29 foram eliminados do processo de estimação do modelo de desempenho via teoria da resposta ao item.




\newpage
\section{Considerações Finais}

Este estudo teve como objetivo mensurar o desempenho (proficiência) dos estudantes de Computação da UFPA no Enade, por meio da TRI. A partir da estimação realizada com ML2P, decorrente da TRI, foi realizada a mensuração da proficiência dos estudantes de Computação que fizeram a prova Enade/2017 e criada uma escala de medida padronizada.

A análise dos itens (questões) evidenciou a capacidade da prova em mensurar a proficiência dos estudantes dos diferentes níveis de domínio cognitivo exigido pelo exame. Nesse contexto, a TRI demonstrou capacidade de capturar a distribuição da proficiência dos estudantes de Computação ao longo dos níveis exigidos pela prova.

Os resultados da pesquisa apontaram que os itens contidos na prova não apresentou nem o domínio cognitivo compreendido pela escala. Este resultado corrobora  o baixo desempenho dos estudantes apontado pelo relatório do Inep (2017) para esta prova e aponta, espeficicamente, em quais aspectos e conhecimentos podem ser econtrados fragilidades de aprendizagem.

Vale ainda destacar que, enquanto estudos realizados com base na TCT abordam a questão do desempenho de maneira agregada, a análise pela TRI permitiu a identificação pontual dos conhecimentos, capacidades e habilidades exigidas em cada nível da escala. Ao desmembrar a proficiência em níveis, evidenciando os conhecimentos exigidos, é possível uma atuação pontual de professores, IES e respectivas autoridades nas questões específicas em que foram demonstradas deficiências na aprendizagem. 

Dados os benefícios mencionados e apresentados pela TRI neste estudo, pesquisas posteriores, relacionadas a determinantes do desempenho de alunos em Computação poderiam adotar como base para mensuração do desempenho dos estudantes as medidas fornecidas por modelos baseados na TRI. Essa análise possibilitaria estudar determinantes baseados em diferentes níveis de proficiência, contribuindo para o avanço nos estudos relacionados a esta área.

\newpage
\section{Referências Bibliográficas}

AKAIKE, H. A new look at the statistical model identification. IEEE Transactions on Automatic
Control., Boston, v.19, n.6, p.716-723, Dec. 1974.

Andrade, D.F., Tavares, H.R., Cunha, R.V. (2000). Teoria da Resposta ao Item: Conceitos e Aplicações. São Paulo: Associação Brasileira de Estatística.

Araújo, R. A. (2019). Análise dos microdados do Enade: proposta de uma ferramenta de exploração utilizando mineração de dados. Dissertação (Mestrado em Ciência da Computação) - Universidade Federal de Goiás, Goiânia.

Birnbaum, A. (1968). Some latent trait models and their use in inferring and examinee's ability. In F.M. Lord \& M.R. Novick, Statistical theories of mental test scores. Reading, MA: Addison-Wesley, ch. 17-20.

Bock, R. D. and Zimowski, M. F. (1997). Multiple Group IRT. In Handbook of Modern Item Response Theory. W.J. van der Linder e R.K. Hambleton Eds. New York: Spring-Verlag.

Baker, F.B.(2001). The basics of item response theory. Washington, DC: ERIC. 

BORGATTO, A. F.; ANDRADE, D. F. de. Análise clássica de testes com diferentes graus de dificuldade. Estudos em Avaliação Educacional, São Paulo, v. 23, n. 52, p. 146–156, 2012.

BRASIL Instituto Nacional de Estudos e Pesquisas Educacionais Anísio Teixeira Sinaes. Brasília, 2017. Disponível em: Disponível em: http://portal.inep.gov.br/web/guest/sinaes Acesso em: 14 março. 2022.

CHARÃO, Andrea S.; WIECHORK, Karina; RODRIGUES, Marlon L. S.; BARBOSA, Fernando P.. Explorando Resultados por Questão no Enade em Ciência da Computação para Subsidiar Revisão de Projeto Pedagógico de Curso. In: WORKSHOP SOBRE EDUCAÇÃO EM COMPUTAÇÃO (WEI), 28. , 2020, Cuiabá. Anais [...]. Porto Alegre: Sociedade Brasileira de Computação, 2020 . p. 16-20. ISSN 2595-6175. DOI: https://doi.org/10.5753/wei.2020.11121.

CUNHA, Renan; SALES, Claudomiro; SANTOS, Reginaldo. Análise Automática com os Microdados do ENADE para Melhoria do Ensino dos Cursos de Ciência da Computação. In: WORKSHOP SOBRE EDUCAÇÃO EM COMPUTAÇÃO (WEI),29., 2021, Evento Online. Anais [...]. Porto Alegre: Sociedade Brasileira de Computação, 2021 . p. 208-217. ISSN 2595-6175. DOI: https://doi.org/10.5753/wei.2021.15912.

Camargo, R. V. W., Camargo, R. de C. C. P., Andrade, D. F. de, \& Bornia, A. C. (2016). Desempenho dos alunos de ciências contábeis na prova ENADE/2012: uma aplicação da Teoria da Resposta ao Item. Revista De Educação E Pesquisa Em Contabilidade (REPeC), 10(3). https://doi.org/10.17524/repec.v10i3.140.

Coelho, E. C., Ribeiro Junior, P. J. \& Bonat, W. H. (2014). Exame nacional de desenvolvimento de estudantes de estatística-desafios e perspectivas pela TRI. Revista da Estatística da Universidade Federal de Ouro Preto, 3(2), pp. 323-337.

\newpage
Chalmers, R., P. (2012). mirt: A Multidimensional Item Response Theory Package for the R Environment. Journal of Statistical Software, 48(6), 1-29. doi: 10.18637/jss.v048.i06.

CHENG, J. et al. Leaflet: Create Interactive Web Maps with the JavaScript ’Leaflet’. Library. 2016. Disponível em: . Acesso em: 03 out. 2016.

CHANG, W. et al. Shiny: Web Application Framework for R. 2017. Disponível em. Acesso em: 10 mar. 2017.

Everitt, B.S, Hothorn (2011). MVA: An Introduction to applied Multivariate Analysis with R.

Hambleton, R.K., Swaminathan, H., Rogers, H.J. (1991). Fundamentals of Item Response Theory. Newburry Park: Sage Publications.

Hair, J., Black, W., Babin, B., Anderson, R. e Tatham, R. (2006) Análise de Dados Multivariados. 6ª Edição, Pearson Prentice Hall, Upper Saddle River.

Joanes, D.N. and Gill, C.A (1998). Comparing measures of sample skewness and kurtosis. The Statistician, 47, 183-189.

KATAOKA, V. Y. et al. O uso do r no ensino de probabilidade na educação básica: Animation e teachingdemos (the use of r in probability teaching at basic education: Animation and teaching demos). Proc. 18o Simpósio Nacional de Probabilidade e Estatística, São Paulo, 2008.

Lawley, D.N. (1943). On problems connected with item selection and test construction. Proceedings of the Royal Society of Edinburgh, Series A, 61, 273-287. 

Lawley, D.N. (1944). The factorial analysis of multiple item tests. Proceedings of the Royal Society of Edinburgh, 62-A, 74-82. 

Lazarsfeld, P.F. (1950). The logical and mathematical foundation of latent structure analysis. In S.A. Stauffer, L. Guttman, E.A. Suchman, P.F. Lazarsfeld, S.A. Star, \& J.A. Clausen (Eds.), Measurement and prediction. Princeton, NJ: Princeton University Press, 1950.

Lord, F.M. (1952). A theory of test scores (Psychometric Monograph No. 7). Iowa City, IA: Psychometric Society.

Lord, F.M. (1953). The relation of test score to the trait underlying the test. Educational and Psychological Measurement, 13, 517-549.

Lord, F.M., Norvick, M.R. (1968). Statistical Theories of Mental Test Score. Reading: Addison-Wesley.

Lord, F.M. (1980). Applications of Item Response Theory to Practical Testing Problems. Hillsdale: Lawrence Erlbaum Associates.

Lima, P. D. S., Ambrosio, A. P., Brancher, J. D., and Felix, I. (2018). Sysenade-analise
das questoes de provas do enade organizadas pelos temas abordados. In Anais dos
Workshops do Congresso Brasileiro de Informatica na Educação, volume 7, page 419.

MARDIA, K. V. Measures of multivariate skewness and kurtosis with applications. Biometrika, 57(3):pp. 519-30, 1970.

MARDIA, K. V. Assessment of multinormality and the robustness of Hotelling s T 2 test. Applied Statistics, London, v. 24, n. 2, p. 163-171, 1975.

NOGUEIRA, S. O. ENADE: Analise de itens de formação geral e de estatística pela TRI. Dissertação de Mestrado, Programa de PósGraduação em Psicologia, Universidade São Francisco, Itatiba, 2008.

PASQUALI, L. Psicometria: teoria dos testes na psicologia e na educação. Petrópolis: Vozes, 2003.

Pestana, MH, \& Gageiro, JG (2003). Análise de dados para ciências sociais: A complementaridade de SPSS [Data Analysis for Social Sciences: The Complementarity of SPSS] (3ª ed.). Lisboa: Edições Silabo.

PASQUALI, L. TRI - Teoria de Resposta ao Item: teoria, procedimentos e aplicações. Brasília: LabPAM/UnB, 2007.

PRIMI, Ricardo et al. Análise do funcionamento diferencial dos itens do Exame Nacional do Estudante (Enade) de Psicologia de 2006. Psico-USF, São Paulo, SP, v. 15, n. 3, p. 379-393, 2010. 

PRIMI, R.; HUTZ, C. S.; SILVA, M. C. R. A prova do ENADE de psicologia 2006: concepção, construção e análise psicométrica da prova. Avaliação Psicológica, v. 10, n. 3, p. 271-294, 2011.

Rasch, G. (1960, 1980). Probabilistic models for some intelligence and attainment tests. Chicago, IL: MESA Press.

Rizopoulos, D. (2006). ltm: An R package for latent variable modelling and item response theory analyses. Journal of Statistical Software, 17(5), 1–25. URL http://www.jstatsoft.org/v17/ i05/.

RABELO, M. Avaliação Educacional: Fundamentos, Metodologia e Aplicações no Contexto Brasileiro. Rio de Janeiro. SBM, 2013.

Revelle, W. (2019). psych: Procedures for Psychological, Psychometric, and Personality Research. Northwestern University, Evanston, Illinois. R package version 1.9.12.

R DEVELOPMENT CORE TEAM. R: A Language and Environment for Statistical Computing, 2022. R Foundation for Statistical Computing. Vienna, Austria. ISBN: 3-90005107-0. 

RSTUDIO. RStudio: Integrated development environment for R (Versão 4.2.2) [Computer software]. Boston, MA. 

Samejima, R. (1969). Estimation of latent ability using a response pattern of graded scores (Psychometric Monograph No. 17). Psychometric Society. 

Samejima, F. (1972). A general model for tree-response data (Psychometric Monograph, No. 18). Psychometric Society. 

SCHWARZ, G. Estimating the dimensional of a model. Annals of Statistics, Hayward, v.6, n.2,
p.461-464, Mar. 1978.

Shipley, Bill. 2004. Cause and Correlation in Biology: A User’S Guide to Path Analysis, Structural Equations and Causal Inference. Vol. 20. 2001.

SCHER, V. T. et al. Uma aplicação da teoria da resposta ao item na avaliação do ENADE do curso de Administração. XIV Colóquio Internacional de Gestão Universitária, Florianópolis-SC, 3 a 5 de dezembro de 2014.

SIEVERT, C. et al. Plotly: Create Interactive Web Graphics via plotly.js. 2021. Disponível em: Acesso em: 03 fev. 2021.

Tucker, L.R. (1946). Maximum validity of a test with equivalent items. Psychometrika, 11, 1-13. 

THIMOTY, A. B. Confirmatory factor analysis for applied research. 2. ed. New York: The Guilford Press, 2015.












\end{document}
